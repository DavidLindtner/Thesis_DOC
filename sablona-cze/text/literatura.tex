% Pro sazbu seznamu literatury použijte jednu z následujících možností

%%%%%%%%%%%%%%%%%%%%%%%%%%%%%%%%%%%%%%%%%%%%%%%%%%%%%%%%%%%%%%%%%%%%%%%%%
%1) Seznam citací definovaný přímo pomocí prostředí literatura / thebibliography

\begin{thebibliography}{99}

    \bibitem{MRS}
MRS UAV System - open source platform for UAV research. \textit{Multi-robot Systems Group} [online]. Praha: Multi-robot Systems Group, c2022 [cit. 2022-01-01]. Dostupné z: \href{http://mrs.felk.cvut.cz/}{http://mrs.felk.cvut.cz/}

    \bibitem{PIX1}
The open standards for drone hardware. \textit{Pixhawk} [online]. San Francisco: Dronecode, 2018 [cit. 2022-01-01]. Dostupné z: \href{https://pixhawk.org/}{https://pixhawk.org/}

    \bibitem{PX4docs}
\textit{PX4 User Guide: PX4 Autopilot User Guide (master)} [online]. San Francisco: Dronecode, 2021 [cit. 2022-03-23]. Dostupné z: \href{https://docs.px4.io/master/en/}{https://docs.px4.io/master/en/}

%     \bibitem{PIX2}
% Pixhawk 4. \textit{PX4} [online]. San Francisco: Dronecode, 2021 [cit. 2022-01-01]. Dostupné z: \href{https://docs.px4.io/master/en/flight\_controller/pixhawk4.html}{https://docs.px4.io/master/en/flight\_controller/pixhawk4.html}

    \bibitem{BetaF}
Betaflight Home: Are you ready to fly?. \textit{Betaflight} [online]. Betaflight, c2022 [cit. 2022-03-25]. Dostupné z: \href{https://betaflight.com/}{https://betaflight.com/}


    \bibitem{ARDU}
Ardupilot, versatile, trusted, open. \textit{Ardupilot} [online]. Ardupilot, c2022 [cit. 2022-01-03]. Dostupné z: \href{https://ardupilot.org/}{https://ardupilot.org/}

    \bibitem{PX4ORG}
PX4: Open Source Autopilot For Drone Developers. \textit{PX4} [online]. San Francisco: Dronecode, c2021 [cit. 2022-01-03]. Dostupné z: \href{https://px4.io/}{https://px4.io/}

	\bibitem{BSDlicense}
The 3-Clause BSD License: BSD-3-Clause. \textit{Open Source Initiative} [online]. West Hollywood: Open Source Initiative, 2014 [cit. 2021-11-17]. Dostupné z: \href{https://opensource.org/licenses/BSD-3-Clause}{https://opensource.org/licenses/BSD-3-Clause}

	\bibitem{INAV}
INAV: Navigation capable flight controller. \textit{GitHub} [online]. California: GitHub, 2021 [cit. 2022-03-27]. Dostupné z: \href{https://github.com/iNavFlight/inav}{https://github.com/iNavFlight/inav}


%     \bibitem{PX4main2}
% PX4 System Architecture. \textit{PX4 Autopilot User Guide} [online]. San Francisco: Dronecode, 2021 [cit. 2021-12-20]. Dostupné z: \href{https://docs.px4.io/master/en/concept/px4\_systems\_architecture.html}{https://docs.px4.io/master/en/concept/px4\_systems\_architecture.html}

	\bibitem{QGround}
QGroundControl. \textit{QGroundControl} [online]. San Francisco: Dronecode, 2019 [cit. 2021-11-24]. Dostupné z: \href{http://qgroundcontrol.com/}{http://qgroundcontrol.com/}

    \bibitem{QGround2}
QGroundControl User Guide. \textit{QGroundControl docs} [online]. San Francisco: Dronecode, 2021 [cit. 2021-12-21]. Dostupné z: \href{https://docs.qgroundcontrol.com/master/en/PlanView/Pattern.html}{https://docs.qgroundcontrol.com/master/en/PlanView/Pattern.html}

    \bibitem{ROS2DDS3}
ROS 2 middleware interface: Mapping between DDS and ROS concepts. \textit{ROS 2 Design} [online]. California: Open Source Robotics Foundation, 2017 [cit. 2021-12-21]. Dostupné z: \href{https://design.ros2.org/articles/ros\_middleware\_interface.html}{https://design.ros2.org/articles/ros\_middleware\_interface.html}

%     \bibitem{UORB1}
% UORB Messaging. \textit{PX4 Autopilot User Guide} [online]. San Francisco: Dronecode, 2021 [cit. 2021-12-21]. Dostupné z: \href{https://docs.px4.io/master/en/middleware/uorb.html}{https://docs.px4.io/master/en/middleware/uorb.html}

%     \bibitem{UORBlist}
% UORB Message Reference. \textit{PX4 Autopilot User Guide} [online]. San Francisco: Dronecode, 2021 [cit. 2021-12-24]. Dostupné z: \href{https://docs.px4.io/master/en/msg\_docs/}{https://docs.px4.io/master/en/msg\_docs/}

%     \bibitem{UORB2}
% RTPS/DDS Interface: PX4-Fast RTPS(DDS) Bridge. \textit{PX4 Autopilot User Guide} [online]. San Francisco: Dronecode, 2021 [cit. 2021-12-21]. Dostupné z: \href{https://docs.px4.io/master/en/middleware/micrortps.html}{https://docs.px4.io/master/en/middleware/micrortps.html}

    \bibitem{CDR}
EProsima Fast Buffers. \textit{EProsima the middleware experts} [online]. Madrid: eProsima, 2014 [cit. 2021-12-23]. Dostupné z: \href{https://www.eprosima.com/docs/fast-buffers/0.3.0/html/index.html}{https://www.eprosima.com/docs/fast-buffers/0.3.0/html/index.html}

    \bibitem{DDS_Standard}
The Real-time Publish-Subscribe Protocol (RTPS) DDS Interoperability Wire Protocol Specification. \textit{Object Management Group} [online]. Milford: OMG, c2021 [cit. 2021-11-25]. Dostupné z: \href{https://www.omg.org/spec/DDSI-RTPS/2.3}{https://www.omg.org/spec/DDSI-RTPS/2.3}
	
	\bibitem{DDS_Def}
Data Distribution Service (DDS). \textit{Object Management Group} [online]. Milford: OMG, 2018 [cit. 2021-11-25]. Dostupné z: \href{https://www.omg.org/omg-dds-portal/}{https://www.omg.org/omg-dds-portal/}

    \bibitem{DDS_Main}
What is DDS? \textit{DDS Foundation: Join DDS Foundation} [online]. Milford: DDS Foundation, c1997-2021 [cit. 2021-11-25]. Dostupné z: \href{https://www.dds-foundation.org/what-is-dds-3/}{https://www.dds-foundation.org/what-is-dds-3/}

    \bibitem{DDS_PubSub}
Introduction to Publish/Subscribe. \textit{Real-Time Innovations: RTI | Inteligent, distributed and real world systems} [online]. Sunnyvale: RTI, c2020 [cit. 2021-11-28]. Dostupné z: \href{https://community.rti.com/static/documentation/connext-dds/6.0.1/doc/manuals/connext\_dds/getting\_started/cpp11/intro\_pubsub\_cpp.html}{https://community.rti.com/static/documentation/connext-dds/6.0.1/doc/manuals/connext\_cpp11/intro\_pubsub\_cpp.html}

    \bibitem{DDS_usage}
Why choose DDS?: Industry standards build on top of DDS. \textit{DDS Foundation: Join DDS Foundation} [online]. Milford: DDS Foundation, c1997-2021 [cit. 2021-11-25]. Dostupné z: \href{https://www.dds-foundation.org/why-choose-dds/}{https://www.dds-foundation.org/why-choose-dds/}

    \bibitem{Eprosima}
RTPS Introduction: What is RTPS? \textit{EProsima: The Middleware experts} [online]. Madrid: eProsima, c2013-2021 [cit. 2021-11-29]. Dostupné z: \href{https://www.eprosima.com/index.php/resources-all/whitepapers/rtps}{https://www.eprosima.com/index.php/resources-all/whitepapers/rtps}

    \bibitem{ROS2DDS}
ROS on DDS: Technical Credibility of DDS. \textit{ROS 2 Design} [online]. California: Open Source Robotics Foundation, c2021 [cit. 2021-11-29]. Dostupné z: \href{https://design.ros2.org/articles/ros\_on\_dds.html}{https://design.ros2.org/articles/ros\_on\_dds.html}


    \bibitem{ROS2DDS2}
About different ROS 2 DDS/RTPS vendors: Supported RMW implementations. \textit{ROS 2 Documentation: Foxy} [online]. California: Open Robotics, c2021 [cit. 2021-12-09]. Dostupné z: \href{https://docs.ros.org/en/foxy/Concepts/About-Different-Middleware-Vendors.html}{https://docs.ros.org/en/foxy/Concepts/About-Different-Middleware-Vendors.html}

%     \bibitem{SIM}
% Simulation. \textit{PX4 Autopilot User Guide} [online]. San Francisco: Dronecode, 2021 [cit. 2021-12-28]. Dostupné z: \href{https://docs.px4.io/master/en/simulation/}{https://docs.px4.io/master/en/simulation/}

    \bibitem{GAZ}
\textit{Gazebo} [online]. California: Open Source Robotics Foundation, c2014 [cit. 2021-12-30]. Dostupné z: \href{http://gazebosim.org/}{http://gazebosim.org/}

    \bibitem{IGN}
THACKSTON, Allison. Ignition vs Gazebo. \textit{Allison Thackston} [online]. San Francisco: Thackston, c2014-2021 [cit. 2021-12-31]. Dostupné z: \href{https://www.allisonthackston.com/articles/ignition-vs-gazebo.html}{https://www.allisonthackston.com/articles/ignition-vs-gazebo.html}

%     \bibitem{INSTALL1}
% Ubuntu Development Environment. \textit{PX4} [online]. San Francisco: Dronecode, 2021 [cit. 2022-01-02]. Dostupné z: \href{https://docs.px4.io/master/en/dev\_setup/dev\_env\_linux\_ubuntu.html}{https://docs.px4.io/master/en/dev\_setup/dev\_env\_linux\_ubuntu.html}

    \bibitem{GRADLE}
How to Install Gradle on Ubuntu 20.04. \textit{Linuxize} [online]. Linuxize, 2020 [cit. 2022-01-02]. Dostupné z: \href{https://linuxize.com/post/how-to-install-gradle-on-ubuntu-20-04/}{https://linuxize.com/post/how-to-install-gradle-on-ubuntu-20-04/}

% \bibitem{ROS2BRIDGE}
% ROS 2 User Guide (PX4-ROS 2 Bridge). \textit{PX4} [online]. San Francisco: Dronecode, 2021 [cit. 2022-01-02]. Dostupné z: \href{https://docs.px4.io/master/en/ros/ros2\_comm.html}{https://docs.px4.io/master/en/ros/ros2\_comm.html}


    \bibitem{ROS2INSTALL}
Installing ROS 2 via Debian Packages. \textit{Documentation: Foxy} [online]. California: Open Robotics, c2021 [cit. 2022-01-02]. Dostupné z: \href{https://docs.ros.org/en/foxy/Installation/Ubuntu-Install-Debians.html}{https://docs.ros.org/en/foxy/Installation/Ubuntu-Install-Debians.html}

%     \bibitem{DDSGEN}
% Fast DDS Installation. \textit{PX4} [online]. San Francisco: Dronecode, 2021 [cit. 2022-01-02]. Dostupné z: \href{https://docs.px4.io/master/en/dev\_setup/fast-dds-installation.html}{https://docs.px4.io/master/en/dev\_setup/fast-dds-installation.html}

    \bibitem{GIT}
PX4\_ros2\_missions. \textit{GitHub} [online]. California: GitHub, 2022 [cit. 2022-01-03]. Dostupné z: \href{https://github.com/DavidLindtner/px4\_ros2\_missions}{https://github.com/DavidLindtner/px4\_ros2\_missions}

    \bibitem{MAVLINK}
\textit{MAVLink: MAVLink Developer Guide} [online]. San Francisco: Dronecode, [2022] [cit. 2022-04-29]. Dostupné z: \href{https://mavlink.io/en/}{https://mavlink.io/en/}


%     \bibitem{OFFBOADR}
% Offboard Mode. \textit{PX4} [online]. San Francisco: Dronecode, 2021 [cit. 2022-01-02]. Dostupné z: \href{https://docs.px4.io/master/en/flight\_modes/offboard.html}{https://docs.px4.io/master/en/flight\_modes/offboard.html}



% \bibitem{sr02/2009}
% 		VUT v~Brně:
%     \emph{Úprava, odevzdávání a zveřejňování vysokoškolských kva\-li\-fi\-kač\-ních prací na VUT v~Brně}\/ [online].
% 		Směrnice rektora č.\,2/2009.
% 		Brno: 2009, po\-sled\-ní aktualizace 24.\,3.\,2009 [cit.\,23.\,10.\,2015].
%     Dostupné z~URL:\\
%     <\url{https://www.vutbr.cz/uredni-deska/vnitrni-predpisy-a-dokumenty/smernice-rektora-f34920/}>.

% \bibitem{CSN_ISO_690-2011}
%     \emph{ČSN ISO 690 (01 0197) Informace a dokumentace -- Pravidla pro bibliografické odkazy a citace informačních zdrojů.}
%     40 stran. Praha: Český normalizační institut, 2011.

% \bibitem{CSN_ISO_7144-1997}
%     \emph{ČSN ISO 7144 (010161) Dokumentace -- Formální úprava disertací a podobných dokumentů.}
%     24 stran. Praha: Český normalizační institut, 1997.

% \bibitem{CSN_ISO_31-11}
%     \emph{ČSN ISO 31-11 Veličiny a jednotky -- část 11: Matematické znaky a značky používané ve fyzikálních vědách a v~technice.}
%     Praha: Český normalizační institut, 1999.

% \bibitem{BiernatovaSkupa2011:CSNISO690komentar}
%     BIERNÁTOVÁ, O., SKŮPA, J.:
%     \emph{Bibliografické odkazy a citace dokumentů dle ČSN ISO 690 (01 0197) platné od 1.\,dubna 2011}\/ [online].
%     2011, poslední aktualizace 2.\,9.\,2011 [cit. 19.\,10.\,2011].
%     Dostupné z~URL:
%     \(<\)\url{http://www.citace.com/CSN-ISO-690.pdf}\(>\)
% %    \(<\)\href{http://www.boldis.cz/citace/citace.html}{http://www.boldis.cz/citace/citace.html}\(>\).

% \bibitem{pravidla}
%     \emph{Pravidla českého pravopisu}.
%     Zpracoval kolektiv autorů. 1.\ vydání.
%     Olomouc: FIN PUB\-LISH\-ING, 1998. 575 s. ISBN 80-86002-40-3.

% \bibitem{Walter1999}
% 	WALTER, G.\,G.; SHEN, X.
% 	\emph{Wavelets and Other Orthogonal Systems}.
% 	2. vyd. Boca Raton: Chapman\,\&\,Hall/CRC, 2000. 392~s. ISBN 1-58488-227-1

% \bibitem{Svacina1999IEEE}
% 	SVAČINA, J.
% 	Dispersion Characteristics of Multilayered Slotlines -- a Simple Approach.
% 	\emph{IEEE Transactions on Microwave Theory and Techniques},
% 	1999, vol.\,47, no.\,9, s.\,1826--1829. ISSN 0018-9480.

% \bibitem{RajmicSysel2002}
%     RAJMIC, P.; SYSEL, P.
%     Wavelet Spectrum Thresholding Rules.
%     In \emph{Proceedings of the International Conference Research in Telecommunication Technology},
%     Žilina: Žilina University, 2002. s.\,60--63. ISBN 80-7100-991-1.

\end{thebibliography}


%%%%%%%%%%%%%%%%%%%%%%%%%%%%%%%%%%%%%%%%%%%%%%%%%%%%%%%%%%%%%%%%%%%%%%%%%
%%2) Seznam citací pomocí BibTeXu
%% Při použití je nutné v TeXnicCenter ve výstupním profilu aktivovat spouštění BibTeXu po překladu.
%% Definice stylu seznamu
%\bibliographystyle{unsrturl}
%% Pro českou sazbu lze použít styl czechiso.bst ze stránek
%% http://www.fit.vutbr.cz/~martinek/latex/czechiso.tar.gz
%%\bibliographystyle{czechiso}
%% Vložení souboru se seznamem citací
%\bibliography{text/literatura}
%
%% Následující příkaz je pouze pro ukázku sazby literatury při použití BibTeXu.
%% Způsobí citaci všech zdrojů v souboru literatura.bib, i když nejsou citovány v textu.
%\nocite{*}