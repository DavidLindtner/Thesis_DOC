\chapter*{Závěr}
\phantomsection
\addcontentsline{toc}{chapter}{Závěr}

Tato diplomová práce se zabývá problematikou simulace misí bezpilotních letadel ve virtuálním prostředí ROS2/Gazebo. V první kapitole jsme popsali topologii dronu z pohledu řídícího systému. Zaobírá se jak řídící jednotkou pro nízkoúrovňové řízení dronu, tak palubním počítačem pro řízení složitých autonomních misí.

Na základě průzkumu trhu s existujícími \textit{open-source} projekty řešícími problematiku řízení bezpilotních letadel ze softwarové stránky jsme zvolili pro další pokračování v práci systém PX4. Vybrali jsme si ho z důvodu profesionálního zaměření a možnosti simulací SITL (\textit{\acl{SITL}}) a HITL (\textit{\acl{HITL}}) a v neposlední řadě z důvodu kvalitního hardwaru - řídící jednotce Pixhawk. V práci je podrobně popsán PX4 firmware, jeho architektura a možnosti propojení a komunikace s ostatními periferiemi. Práce taky pojednává o důležitém článku ekosystému PX4, a tím je software QGroundControl, software pro vzdálenou kontrolu letu, plánování různých typů bezpilotních misí a nastavení vnitřních a vnějších parametrů dronu.

Důležitým bodem práce bylo prozkoumat možnosti komunikace mezi ROS 2 a PX4 firmware. Práce se nejdříve ubírala směrem komunikace pomocí Fast RTPS bridge. Rozebírá se zde samotný \acs{DDS} (\acl{DDS}) middleware a jeho komunikační protokol \acs{RTPS} (\acl{RTPS}) pro kritické aplikace. Klade se zde důraz na základní vlastnosti \acs{DDS} a hlavně propojení mezi ROS 2 a \acs{DDS} komunikačním middlewarem. V průběhu práce jsme zjistili, že komunikace prostřednictvím Fast \acs{RTPS} (\acs{DDS} middleware) bude velmi vhodný přístup v budoucnu, ale na základě chybějící dokumentace ke zprávám \textit{uORB} (zprávy vnitřní komunikace v PX4 systémech) je velice obtížné posílat povely systému PX4 přímo pomocí Fasst RTPS protokolu (\acs{DDS} middlewaru). Dalším problémem v tomto modelu komunikace je, že systém PX4 v této době neumožňuje komunikaci s více bezpilotními letadly najednou pomocí Fast RTPS protokolu. Z toho důvodu jsme v práci pokračovali s ověřeným komunikačním protokolem MAVLink, který je implementovaný ROS 2 uzlem Mavros.

Z důvodu opakovatelnosti výsledků práce jsme značnou část práce věnovali popisu instalace a nastavení všech programů a závislostí potřebných k vytvoření virtuálního prostředí pro simulaci implementovaných robotických misí.

Poslední kapitola popisuje návrh robotických misí ve virtuálním ekosystému PX4 - Gazebo - ROS 2. V práci byli implementovány autonomní mise pro let na globální a lokální waypointy, let podle setpointů rychlosti a mise pro sledování dynamického objektu. V práci byly taky testovány autonomní letecké mise s více drony najednou. Všechny úkony autonomních misí byly ovládány z nadřazeného uzlu a kromě vzletu a přistání byl v dronu aktivován tzv. mimopalubní (\textit{offboard}) letový režim. Všechny mise byly realizovány v rámci ROS 2 balíčku a zdrojové kódy jsou k dispozici v přílohách práce.

Přidaná hodnota práce je v tom, že vytvořené řešení je snadno přenositelné ze simulovaného prostředí na reálný hardware.
