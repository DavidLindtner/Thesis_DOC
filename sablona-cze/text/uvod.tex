\chapter*{Úvod}
\phantomsection
\addcontentsline{toc}{chapter}{Úvod}

DOKONČIŤ

Téma bezpilotních letadel a autonomních leteckých misí v poslední době nabývá na popularitě. S drony se začínáme setkávat v běžném životě, ale také v dalších odvětvích, jako jsou armáda, lesnictví, zemědělství a stavebnictví. Simulace je vhodným přístupem k vývoji autonomních letadel, protože snižuje finanční nároky a urychluje vývoj a testování.

Tato práce se bude zabývat problematikou simulace misí bezpilotních letadel ve virtuálním prostředí. Cílem práce je seznámit se se simulačním ekosystémem Gazebo/ROS 2 a vytvořit a demonstrovat jednoduchou autonomní misi.

Práce bude pojednávat o topologii řídícího systému dronu, jak ze stránky nízkoúrovňového řízení, tak z pohledu palubního počítače pro řízení složitých autonomních misí.

V práci bude popsaný firmware řídící jednotky, možnosti nastavení parametrů letu dronu, možnosti komunikace s okolím a hlavně komunikace s ROS 2.

Práce se věnuje instalaci a nastavení simulačního prostředí. Bude zde přiblížen praktický test simulované autonomní mise v prostředí PX4 - Gazebo, v které dron dostává povely z nadřazeného systému palubního počítače na báze ROS 2.