\chapter*{Úvod}
\phantomsection
\addcontentsline{toc}{chapter}{Úvod}

Téma bezpilotních letadel a autonomních leteckých misí v poslední době nabývá na popularitě. S drony se začínáme setkávat v běžném životě, ale také v dalších odvětvích, jako jsou armáda, lesnictví, zemědělství a stavebnictví. Simulace je vhodným přístupem k vývoji softwaru pro autonomní letadla, protože snižuje finanční nároky a urychluje vývoj a testování.

Tato práce se bude zabývat problematikou simulace robotických misí bezpilotních letadel ve virtuálním prostředí. Cílem práce je seznámit se se simulačním ekosystémem Gazebo/ROS 2 a vytvořit a demonstrovat autonomní letecké mise. Důraz bude hlavně kladen na stabilitu softwarového řešení a stabilitu komunikace mezi kritickými komponenty robotické letecké mise.

Budeme se zaměřovat na letecké mise jako je let po globálních nebo lokálních waypointech a sledování dynamického objektu. Bude zde rozebráno složitější řízení dronu pomocí lineárních, nebo úhlových rychlostí. 

Práce bude pojednávat o topologii řídícího systému dronu, jak ze stránky nízkoúrovňového řízení pohonů a stabilizace, tak z pohledu palubního počítače pro řízení komplexních autonomních leteckých misí.

V práci bude popsaný firmware řídící jednotky, možnosti nastavení parametrů letu dronu, možnosti komunikace s okolím a hlavně komunikace s ROS 2 pomocí \acs{RTPS} (\acl{RTPS}) nebo protokolu Mavlink.

Všechny softwarové komponenty budeme volit s ohledem na to, aby bylo vytvořené řešení snadno přenositelné z virtuálního prostředí na reálný hardware.

Aby byly výsledky práce opakovatelné, tak se zaměříme na podrobný popis instalace a nastavení všech programů a závislostí potřebných k spuštění simulace. Dále zde bude přiblížen praktický test simulované autonomní mise v prostředí PX4 - Gazebo pro jedno i více bezpilotních letadel. Simulované drony zde budou dostávat povely z nadřazeného systému palubního počítače na báze ROS 2.
